\documentclass[dvipdfmx]{jsarticle}
\usepackage[dvipdfmx]{graphicx}

\begin{document}
\title{認識工学第9回小レポート課題}
\author{19C1123 横尾陸}
\date{\today}
\maketitle
\section*{問題}
スカラー量(ex. N人の身長)に対しては、適当な階級の幅を設定して「ヒストグラム」を作ることで簡単な量子化(データを複数のグループに分けること)ができる。これに対して、ベクトル量(ex. 音声のケプストラム特徴量)の場合も、各次元に対して同様に階級の幅を設定してヒストグラムのようなグループ分けをすれば良さそうに見えるが、実はそれはあまり賢い方法ではい。なぜそれではうまくいかないかの理由を、動画その1で説明した「クラスタリング」と「ベクトル量子化」に関連付けて説明せよ。\\

ベクトル量をクラスタリングする際にほとんどの場所にデータがないということが起きる.データを予めみてデータの設定数の部分空間に分割する必要があり,分類したところに番号(シンボル)をつけるがその番号の大小は関係なく,背番号的な役割であり,セントロイド(代表点)の座標を決定しコードブックに格納する.コードブックは分類した特徴点が記録されており,ベクトル量子化の際にそのコードブックを用いる.ヒストグラムの一区切り一区切りは同一間隔で行う.しかしベクトル量はほとんどの場所にデータがなくある一部にデータが集まっており1分割の空間もまばらであるためにヒストグラムで表したとしてもあまり意味がなくなってしまう.


\end{document}

