\documentclass[dvipdfmx]{jsarticle}
\usepackage[dvipdfmx]{graphicx}

\begin{document}
\title{認識工学第3回課題}
\author{19C1123 横尾陸}
\date{\today}
\maketitle

\section{$p(W|X)$と$p(X|W)$の違いを述べよ.}
$p(W|X)$は$X$という条件で起こる$W$が起こる条件付き確率のこと.\\
$p(X|W)$は$W$という条件で起こる$X$が起こる条件付き確率である.

\section{式変形の根拠を述べよ.}
$\tilde{W}=argmax \: p(W|X)=argmax \: p(X|W)p(W)$を考える.\\
条件付き確率は$p(W|X)=\frac{p(X\cap W)}{p(X)}$で求められる.\\
乗法の定理$p(X\cap W)=p(W)\times p(W|X)$を使用し,
$$p(W|X)=\frac{p(X\cap W)}{p(X)}=\frac{p(X|W)p(W)}{p(X)}$$
となる.
ここでは$W$について考えているので分母の$p(X)$は無視できるので
$$\tilde{W}=argmax\:p(W|X)=argmax\:p(X|W)p(W)$$
となる.

\end{document}

