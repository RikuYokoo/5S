\documentclass[dvipdfmx]{jsarticle}
\usepackage[dvipdfmx]{graphicx}

\begin{document}
\title{認識工学第6回レポート課題}
\author{19C1123 横尾陸}
\date{\today}
\maketitle
\section*{問題}
いま我々が用いている自然言語音声(日本語や英語)は,進化の過程で何に対して最適化されてきたと思われるか,考察せよ.
\section*{考察}
進化の過程でコミュニケーション(情報の伝達)に対して最適化されてきたと思われる.動物は言葉を話せなくても鳴き声などの声を発することはできる.その声の声色で大体の感情を理解することはできる.しかし声に意味をもたせることでお互いの考え,感情などの伝達が伝わりやすくなる.一声の情報量が多くなる.人間は集団で生活をしてきたので効率的な情報交換の手段が必要だったと考えるとコミュニケーションに対して最適化されてきたと考えることができる.



\end{document}

