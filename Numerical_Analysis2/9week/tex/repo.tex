\documentclass[dvipdfmx]{jsarticle}
\usepackage[dvipdfmx]{graphicx}

\begin{document}
\title{数値解析学第9回小レポート課題}
\author{19C1123 横尾陸}
\date{\today}
\maketitle

\section*{問題}
講義動画その3で用いた例題3(コインを8回投げて裏が7回・・・)に関して,  
(1) この場合の確率を計算せよ.
8回投げるから$\left(\frac{1}{2}\right)^8$となり,そのうち裏が7回なので${}_8 C_7$で$$\left(\frac{1}{2}\right)^8\times{}_8 C_7=\frac{1}{32}$$となる.よって確率は3.125\%となる.\\
(2)コインを投げる回数が20回になったとき、例題と同じ条件で帰無仮説が棄却されるのは裏の枚数が何枚以下になった場合か、計算せよ。\\
帰無仮説が棄却される確率は5\%とする.\\
裏が6回の時の確率は
$$\left(\frac{1}{2}\right)^{20}\times{}_{20} C_6 = 0.03696$$
より確率は$3.7\%$である.\\
裏が7回のときの確率は
$$\left(\frac{1}{2}\right)^{20}\times{}_{20} C_7 = 0.07392$$
より確率は$7.4\%$である.
よって帰無仮説が棄却されるのは裏の枚数が6枚以下になった場合である.



\end{document}

