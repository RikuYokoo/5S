\documentclass[dvipdfmx]{jsarticle}
\usepackage[dvipdfmx]{graphicx}

\begin{document}
\title{数値解析学小レポート課題}
\author{19C1123 横尾陸}
\date{\today}
\maketitle

\section*{課題}
もしそのように記録された映像や音声が手元にあるとして、たとえば、ずっと後になってから「これまでに見た海の景色を振り返りながら順に見たい」「昨年の夏頃にあの人と会ったときの会話をもう一度聞きたい」のようなことを実現するためには、どのような技術が必要か、\\

記録媒体に保存しているデメリットとしていつ撮ったのかということがわかりにくいということだ.
その日ごと区分けるのはもちろんどんな場面なのかどこに行った,などが軽く書かれているとそうだったとなりわかる.あとは情景を入力するとそれが写っている場面を出してくれる検索機能などがあったらいいと思う.
機械学習の分類を使用すれば情景の分類などができると考えた.過去から振り返るということからソート機能も必要である.

\end{document}

