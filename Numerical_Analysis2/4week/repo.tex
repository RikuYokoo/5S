\documentclass[dvipdfmx]{jsarticle}
\usepackage[dvipdfmx]{graphicx}

\begin{document}
\title{数値解析学2第4回小レポート}
\author{19C1123 横尾陸}
\date{\today}
\maketitle

\section*{機械学習について知っていること,興味があること}
機械学習の中にもいろいろな手法がある.教師あり学習,教師なし学習,強化学習,深層学習などがある.教師あり学習は正解データを与えてそれに近づくように学習が進んでいく.教師なし学習はデータのパターンを学習させるという認識をしている.強化学習はyoutubeなどで強化学習をやっているのを見たことがあるが歩行を0からできるようにさせたり,多関節の棒状のやつに前に進むことを強化学習でやらせたりしているのを見たことがある.深層学習はニューラルネットワークと言われる層を多くすることで実現している.

機械学習をやるためのフレームワークとして,TensorFlow,pytorch,今はサポートが終了してしまったがchainerなどがある.
darknet,yoloはリアルタイム物体検出で広く用いられている.環境さえ整えば簡単?に物体検出ができる.

強化学習に興味がある.強化学習について詳しく知らないがイメージで勝手に?ゴールに向かって学習していて結果の道筋(歩くことの強化学習なら歩き方がどうなるかわからない)がどうなるか終わるまでわからないイメージがある.

\section*{授業の感想・要望}
授業動画の音声について,機械の合成音ではなく実際の声のほうがよいと思った.機械の合成音だと授業という気がしないと感じた.

おすすめの機械学習のツール,シュミレーション環境が知りたいです.

\end{document}
