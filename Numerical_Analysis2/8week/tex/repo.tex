\documentclass[dvipdfmx]{jsarticle}
\usepackage[dvipdfmx]{graphicx}

\begin{document}
\title{第8回小レポート課題}
\author{19C1123 横尾陸}
\date{\today}
\maketitle
\section*{問題}
(問題2)講義動画その3で紹介した「調音のしくみ」に関連して、「早口言葉」はなぜ喋りにくいのか、「竹垣に竹立てかけた」を例として考察してください。

'ta'という発音は舌を口の上につけ言葉を発するときに離すことで声を出している.授業でも言っていたが舌は成人で150g~200gくらいあり早口言葉を離すときはこの重さのものを高速に動かして話している.「竹垣に竹立てかけた」という言葉は12文字中4文字"ta"の発音があり言うたびに舌を口の上につけることをしなければいけないので言いづらくなる.「たけ」や「たて」など母音が同じ言葉が続くので頭でしっかり意識しないと言い間違えてしまうのではないかと考えた.
た行は特に舌を動かし,口を他の言葉を発するときより開かないと聞き取りづらくなってしまうので無意識による歯との接触を避ける行為によって発音しづらくなっていると考えた.


\end{document}

